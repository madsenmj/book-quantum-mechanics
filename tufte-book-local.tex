
\hypersetup{colorlinks}
\usepackage{microtype}
\usepackage{booktabs} % Better horizontal rules in tables
\usepackage{afterpage}
\usepackage{verbatim}
\usepackage{bbm}

\usepackage{framed}

%packages for drawing figures.
\usepackage{tikz}
\usepackage{pgf}
\usepackage{sagetex}
\usepackage{tikzscale}
\usetikzlibrary{snakes,calc,3d,shapes,intersections,fadings,decorations.pathreplacing}
\usepackage{tikz-3dplot}


%makes sure tikz plot uses the bounds to plot.
\def\tikz@plot@samples@recalc#1:#2\relax{%
  \pgfmathparse{#1}%
  \let\tikz@temp@start=\pgfmathresult%
  \pgfmathparse{#2}%
  \let\tikz@temp@end=\pgfmathresult%
  \pgfmathparse{\tikz@temp@start+(\tikz@temp@end-\tikz@temp@start)/(\tikz@plot@samples-1)}%
  \edef\tikz@plot@samplesat{\tikz@temp@start,\pgfmathresult,...,\tikz@temp@end,\tikz@temp@end}%
}

\usepackage{graphicx} % Needed to insert images into the document
\graphicspath{{graphics/}{chapter01/figures/}} % Sets the default location of pictures


\setkeys{Gin}{width=\linewidth,totalheight=\textheight,keepaspectratio} % Improves figure scaling


\usepackage{fancyvrb} % Allows customization of verbatim environments
\fvset{fontsize=\normalsize} % The font size of all verbatim text can be changed here

\newcommand{\hangp}[1]{\makebox[0pt][r]{(}#1\makebox[0pt][l]{)}} % New command to create parentheses around text in tables which take up no horizontal space - this improves column spacing
\newcommand{\hangstar}{\makebox[0pt][l]{*}} % New command to create asterisks in tables which take up no horizontal space - this improves column spacing

\usepackage{xspace} % Used for printing a trailing space better than using a tilde (~) using the \xspace command

\newcommand{\monthyear}{\ifcase\month\or January\or February\or March\or April\or May\or June\or July\or August\or September\or October\or November\or December\fi\space\number\year} % A command to print the current month and year

\newcommand{\openepigraph}[2]{ % This block sets up a command for printing an epigraph with 2 arguments - the quote and the author
\begin{fullwidth}
\sffamily\large
\begin{doublespace}
\noindent\allcaps{#1}\\ % The quote
\noindent\allcaps{#2} % The author
\end{doublespace}
\end{fullwidth}
}

\newcommand{\blankpage}{\newpage\hbox{}\thispagestyle{empty}\newpage} % Command to insert a blank page

\usepackage{units} % Used for printing standard units

\newcommand{\hlred}[1]{\textcolor{Maroon}{#1}} % Print text in maroon
\newcommand{\hangleft}[1]{\makebox[0pt][r]{#1}} % Used for printing commands in the index, moves the slash left so the command name aligns with the rest of the text in the index
\newcommand{\hairsp}{\hspace{1pt}} % Command to print a very short space
\newcommand{\ie}{\textit{i.\hairsp{}e.}\xspace} % Command to print i.e.
\newcommand{\eg}{\textit{e.\hairsp{}g.}\xspace} % Command to print e.g.
\newcommand{\na}{\quad--} % Used in tables for N/A cells
\newcommand{\measure}[3]{#1/#2$\times$\unit[#3]{pc}} % Typesets the font size, leading, and measure in the form of: 10/12x26 pc.
\newcommand{\tuftebs}{\symbol{'134}} % Command to print a backslash in tt type in OT1/T1

\providecommand{\XeLaTeX}{X\lower.5ex\hbox{\kern-0.15em\reflectbox{E}}\kern-0.1em\LaTeX}
\newcommand{\tXeLaTeX}{\XeLaTeX\index{XeLaTeX@\protect\XeLaTeX}} % Command to print the XeLaTeX logo while simultaneously adding the position to the index

\newcommand{\doccmdnoindex}[2][]{\texttt{\tuftebs#2}} % Command to print a command in texttt with a backslash of tt type without inserting the command into the index

\newcommand{\doccmddef}[2][]{\hlred{\texttt{\tuftebs#2}}\label{cmd:#2}\ifthenelse{\isempty{#1}} % Command to define a command in red and add it to the index
{ % If no package is specified, add the command to the index
\index{#2 command@\protect\hangleft{\texttt{\tuftebs}}\texttt{#2}}% Command name
}
{ % If a package is also specified as a second argument, add the command and package to the index
\index{#2 command@\protect\hangleft{\texttt{\tuftebs}}\texttt{#2} (\texttt{#1} package)}% Command name
\index{#1 package@\texttt{#1} package}\index{packages!#1@\texttt{#1}}% Package name
}}

\newcommand{\doccmd}[2][]{% Command to define a command and add it to the index
\texttt{\tuftebs#2}%
\ifthenelse{\isempty{#1}}% If no package is specified, add the command to the index
{%
\index{#2 command@\protect\hangleft{\texttt{\tuftebs}}\texttt{#2}}% Command name
}
{%
\index{#2 command@\protect\hangleft{\texttt{\tuftebs}}\texttt{#2} (\texttt{#1} package)}% Command name
\index{#1 package@\texttt{#1} package}\index{packages!#1@\texttt{#1}}% Package name
}}

% A bunch of new commands to print commands, arguments, environments, classes, etc within the text using the correct formatting
\newcommand{\docopt}[1]{\ensuremath{\langle}\textrm{\textit{#1}}\ensuremath{\rangle}}
\newcommand{\docarg}[1]{\textrm{\textit{#1}}}
\newenvironment{docspec}{\begin{quotation}\ttfamily\parskip0pt\parindent0pt\ignorespaces}{\end{quotation}}
\newcommand{\docenv}[1]{\texttt{#1}\index{#1 environment@\texttt{#1} environment}\index{environments!#1@\texttt{#1}}}
\newcommand{\docenvdef}[1]{\hlred{\texttt{#1}}\label{env:#1}\index{#1 environment@\texttt{#1} environment}\index{environments!#1@\texttt{#1}}}
\newcommand{\docpkg}[1]{\texttt{#1}\index{#1 package@\texttt{#1} package}\index{packages!#1@\texttt{#1}}}
\newcommand{\doccls}[1]{\texttt{#1}}
\newcommand{\docclsopt}[1]{\texttt{#1}\index{#1 class option@\texttt{#1} class option}\index{class options!#1@\texttt{#1}}}
\newcommand{\docclsoptdef}[1]{\hlred{\texttt{#1}}\label{clsopt:#1}\index{#1 class option@\texttt{#1} class option}\index{class options!#1@\texttt{#1}}}
\newcommand{\docmsg}[2]{\bigskip\begin{fullwidth}\noindent\ttfamily#1\end{fullwidth}\medskip\par\noindent#2}
\newcommand{\docfilehook}[2]{\texttt{#1}\index{file hooks!#2}\index{#1@\texttt{#1}}}
\newcommand{\doccounter}[1]{\texttt{#1}\index{#1 counter@\texttt{#1} counter}}

\usepackage{makeidx} % Used to generate the index
\makeindex % Generate the index which is printed at the end of the document

% This block contains a number of shortcuts used throughout the book
\newcommand{\TL}{Tufte-\LaTeX\xspace}
\newcommand{\CAS}{C.A.S.\xspace}


% add numbers to chapters, sections
\setcounter{secnumdepth}{1}

\usepackage{xcolor} % for colour
\usepackage{lipsum} % just for sample text

% part format
\titleclass{\part}{top}
\titleformat{\part}%
  {\huge\rmfamily\itshape\color{green}}% format applied to label+text
  {\llap{\colorbox{green}{\parbox{2.5cm}{\hfill\itshape\huge\color{white} Part \thepart}}}}% label
  {2pt}% horizontal separation between label and title body
  {}% before the title body
  []% after the title body

% chapter format
\titleformat{\chapter}%
  {\huge\rmfamily\itshape\color{blue}}% format applied to label+text
  {\llap{\colorbox{blue}{\parbox{1.5cm}{\hfill\itshape\huge\color{white}\thechapter}}}}% label
  {2pt}% horizontal separation between label and title body
  {}% before the title body
  []% after the title body

% section format
\titleformat{\section}%
  {\normalfont\Large\itshape\color{orange}}% format applied to label+text
  {\llap{\colorbox{orange}{\parbox{1.5cm}{\hfill\color{white}\thesection}}}}% label
  {1em}% horizontal separation between label and title body
  {}% before the title body
  []% after the title body

% subsection format
\titleformat{\subsection}%
  {\normalfont\large\itshape\color{red}}% format applied to label+text
  {\llap{\colorbox{red}{\parbox{1.5cm}{\hfill\color{white}\thesubsection}}}}% label
  {1em}% horizontal separation between label and title body
  {}% before the title body
  []% after the title body


\usepackage{amsmath}
\usepackage{amsfonts}
\usepackage{amssymb}

% The changebar package allows colored bars along the text
% edge for setting off examples and derivations.  However,
% it also puts bars next to footnotes, which is undesired
% for this class, so we "undo" that functionality with the
% following hack
\let\textbookcls@footnotetext\@footnotetext
\let\textbookcls@mpfootnotetext\@mpfootnotetext

\usepackage{ifpdf}
\usepackage{tocloft}

\usepackage{hyperref}
\numberwithin{equation}{chapter}
\usepackage{cleveref}

%this should fix the spsp pages at the end due to changebar
%\makeatletter
\AtEndDocument{%
\def\cb@checkPdfxy#1#2#3#4#5{%
\cb@@findpdfpoint{#1}{#2}%
\ifdim#3sp=\cb@pdfx\relax      % <<-- original has \ifnum#3=\cb@pdfx
\ifdim#4sp=\cb@pdfy\relax      % <<-- original has \ifnum#4=\cb@pdfy
\ifdim#5=\cb@pdfz\relax
\else
\cb@error
\fi
\else
\cb@error
\fi
\else
\cb@error
\fi
}}
%\makeatother


\usepackage[leftbars,color]{changebar}

\let\@footnotetext\textbookcls@footnotetext
\let\@mpfootnotetext\textbookcls@mpfootnotetext


 \setlength{\changebarsep}{-1pt}
 \setlength{\changebarwidth}{2pt}
 \definecolor{BarColor}{rgb}{.1,.3,.6}
 \definecolor{ARColor}{rgb}{.5,.0,.5}
 \definecolor{ToolColor}{rgb}{1,0,0}

% \definecolor{DeriveColor}{rgb}{.8,.3,.6}
 \definecolor{DeriveColor}{rgb}{.85,.2,.25}
 \definecolor{ExColor}{rgb}{0.1,.6,.12}
 \definecolor{TBColor}{rgb}{0.9,.9,0.0}
 \cbcolor{BarColor}

%example environment
\newcounter{example}[chapter]
\renewcommand{\theexample}{\thechapter.\arabic{example}}

\ifpdf
    \renewcommand{\theHexample}{\thechapter.\arabic{example}} %This is needed to keep unique links for hyperref
\fi

\newenvironment{example}
    {\cbcolor{BarColor}\par \vspace{10pt} \cbstart
    \begin{enumerate}\item[]
     \refstepcounter{example}\noindent\color{BarColor}\large\textbf{Example
     \theexample} \normalsize\color{black} } {\end{enumerate} \cbend{}\par \vspace{8pt}}

%-----------------------------------------------------------
% Active reader margin note environmnet arnote{}
%-----------------------------------------------------------

\newcommand*{\@tufte@arnote@justification}{\@tufte@justification@autodetect}
\define@choicekey*+[tufte]{common}{arnote}[\@tufte@kvtext\@tufte@kvnum]{justified,raggedleft,raggedright,raggedouter,auto}[auto]{%
  \ifcase\@tufte@kvnum\relax
    \renewcommand*{\@tufte@arnote@justification}{\justifying}% justified
  \or
    \renewcommand*{\@tufte@arnote@justification}{\RaggedLeft}% ragged left
  \or
    \renewcommand*{\@tufte@arnote@justification}{\RaggedRight}% ragged right
  \or
    \renewcommand*{\@tufte@arnote@justification}{\@tufte@justification@outer}% ragged outer (flush right on verso pages, flush left on recto pages)
  \or
    \renewcommand*{\@tufte@arnote@justification}{\@tufte@justification@autodetect}% autodetects best justification mode based on all class options
  \fi
}{%
  \TufteWarningNL{Invalid option `#1' for arnote key.\MessageBreak Must be one of: justified, raggedleft,\MessageBreak raggedright, raggedouter, auto}
  \renewcommand*{\@tufte@arnote@justification}{\@tufte@justification@autodetect}% autodetects best justification mode based on all class options
}

%-----------------------------------------------------------
% Tool margin note environmnet toolnote{}
%-----------------------------------------------------------

\newcommand*{\@tufte@toolnote@justification}{\@tufte@justification@autodetect}
\define@choicekey*+[tufte]{common}{arnote}[\@tufte@kvtext\@tufte@kvnum]{justified,raggedleft,raggedright,raggedouter,auto}[auto]{%
  \ifcase\@tufte@kvnum\relax
    \renewcommand*{\@tufte@toolnote@justification}{\justifying}% justified
  \or
    \renewcommand*{\@tufte@toolnote@justification}{\RaggedLeft}% ragged left
  \or
    \renewcommand*{\@tufte@toolnote@justification}{\RaggedRight}% ragged right
  \or
    \renewcommand*{\@tufte@toolnote@justification}{\@tufte@justification@outer}% ragged outer (flush right on verso pages, flush left on recto pages)
  \or
    \renewcommand*{\@tufte@toolnote@justification}{\@tufte@justification@autodetect}% autodetects best justification mode based on all class options
  \fi
}{%
  \TufteWarningNL{Invalid option `#1' for arnote key.\MessageBreak Must be one of: justified, raggedleft,\MessageBreak raggedright, raggedouter, auto}
  \renewcommand*{\@tufte@toolnote@justification}{\@tufte@justification@autodetect}% autodetects best justification mode based on all class options
}

% The ``marginals'' key simultaneously sets the same justification for all marginal material
\define@choicekey*+[tufte]{common}{marginals}[\@tufte@kvtext\@tufte@kvnum]{justified,raggedleft,raggedright,raggedouter,auto}[auto]{%
  \ifcase\@tufte@kvnum\relax
    \ExecuteOptionsX[tufte]<common>{citation=justified,sidenote=justified,caption=justified,marginnote=justified,arnote=justified,toolnote=justified}% justified
  \or
    \ExecuteOptionsX[tufte]<common>{citation=raggedleft,sidenote=raggedleft,caption=raggedleft,marginnote=raggedleft,arnote=raggedleft,toolnote=raggedleft}% ragged left
  \or
    \ExecuteOptionsX[tufte]<common>{citation=raggedright,sidenote=raggedright,caption=raggedright,marginnote=raggedright,arnote=raggedright,toolnote=raggedright}% ragged right
  \or
    \ExecuteOptionsX[tufte]<common>{citation=raggedouter,sidenote=raggedouter,caption=raggedouter,marginnote=raggedouter,arnote=raggedouter,toolnote=raggedouter}% ragged outer (flush right on verso pages, flush left on recto pages)
  \or
    \ExecuteOptionsX[tufte]<common>{citation=auto,sidenote=auto,caption=auto,marginnote=auto,arnote=auto,toolnote=auto}% autodetects best justification mode based on all class options
  \fi
}{%
  \TufteWarningNL{Invalid option `#1' for marginals key.\MessageBreak Must be one of: justified, raggedleft,\MessageBreak raggedright, raggedouter, auto}
  \ExecuteOptionsX[tufte]<common>{citation=auto,sidenote=auto,caption=auto,marginnote=auto,arnote=auto,toolnote=auto}% autodetects best justification mode based on all class options
}

\newcommand*{\@tufte@arnote@font}{\@tufte@marginfont}
\newcommand*{\setarnotefont}[1]{\renewcommand*{\@tufte@arnote@font}{#1}}
\newcommand*{\@tufte@toolnote@font}{\@tufte@marginfont}
\newcommand*{\settoolnotefont}[1]{\renewcommand*{\@tufte@toolnote@font}{#1}}


% Same format as marginnote
\newcounter{actread}[chapter]
\renewcommand{\theactread}{\thechapter.\arabic{actread}}
\usepackage{ifpdf}
\ifpdf
    \renewcommand{\theHactread}{\thechapter.\arabic{actread}} %This is needed to keep unique links for hyperref
\fi
\newcommand\arnote[2][0pt]{%
  \let\cite\@tufte@infootnote@cite%   use the in-sidenote \cite command
  \gdef\@tufte@citations{}%           clear out any old citations
  \@tufte@margin@par%                 use parindent and parskip settings for marginal text
  \marginpar{\hbox{}\vspace*{#1}\@tufte@arnote@font\@tufte@arnote@justification\vspace*{-1\baselineskip}\noindent
   \refstepcounter{actread}\color{ARColor}\large\textbf{Active Reading \theactread :} \normalsize\color{black} #2 }%
  \@tufte@reset@par%                  use parindent and parskip settings for body text
  \@tufte@print@citations%            print any citations
  \let\cite\@tufte@normal@cite%       go back to using normal in-text \cite command
}


\newcommand\toolnote[2][0pt]{%
  \let\cite\@tufte@infootnote@cite%   use the in-sidenote \cite command
  \gdef\@tufte@citations{}%           clear out any old citations
  \@tufte@margin@par%                 use parindent and parskip settings for marginal text
  \marginpar{\hbox{}\vspace*{#1}\@tufte@arnote@font\@tufte@arnote@justification\vspace*{-1\baselineskip}\noindent \color{ToolColor}\normalsize\textbf{Quantum Mechanic's Toolbox: } \normalsize\color{black} #2}%
  \@tufte@reset@par%                  use parindent and parskip settings for body text
  \@tufte@print@citations%            print any citations
  \let\cite\@tufte@normal@cite%       go back to using normal in-text \cite command
}




\newcommand{\exProblem}{\par\vspace{2pt}\small\noindent\noindent}
\newcommand{\exSolution}{\par\vspace{10pt}\noindent\textbf{Solution:} }

\newenvironment{derivation}[1]
    {\cbcolor{DeriveColor}\par \vspace{10pt}
     \cbstart
    \begin{enumerate}\item[]
     \noindent\textbf{#1}
     \par\vspace{5pt}\small\noindent
    } {\end{enumerate} \cbend\par \vspace{10pt}}


%-----------------------------------------------------------
% exercise environment
%-----------------------------------------------------------

\newcounter{exercise}[chapter]
\renewcommand{\theexercise}{\thechapter.\arabic{exercise}}

\ifpdf
    \renewcommand{\theHexample}{\thechapter.\arabic{exercise}} %This is needed to keep unique links for hyperref
\fi

\newenvironment{exercise}
    {
    %\noindent\textcolor{ExColor}{\rule{\textwidth}{2pt}}\par
    \cbcolor{ExColor}\par \vspace{10pt} \cbstart
    \begin{enumerate}\item[]
     \refstepcounter{exercise}\noindent\color{ExColor}\large\textbf{Exercise
     \theexercise} \normalsize\color{black} } {\end{enumerate}
      \cbend{}\par \vspace{8pt} }%\noindent\textcolor{ExColor}{\rule{\textwidth}{2pt}}\par



%-----------------------------------------------------------
% Tactics Box environment
%-----------------------------------------------------------

\newcounter{tactics}[chapter]
\renewcommand{\thetactics}{\thechapter.\arabic{tactics}}

\newenvironment{tacticsbox}{%
   \def\FrameCommand{\colorbox{TBColor}}%
   \MakeFramed{\advance\hsize-\width \FrameRestore \refstepcounter{tactics} \noindent \textbf{Technique
     \thetactics:}}}
 {\endMakeFramed}


\newcommand{\ExSection}[1][ ]{
    \clearpage
    \section*{Exercises}
    \markright{Exercises}
    \addcontentsline{toc}{section}{Exercises}
    \WriteChap{Chapter \thechapter\ Solutions}
    }

%\newenvironment{exercises}[1]
%    {\noindent \subsubsection[Exercises]{\emph{Exercises for \ref{#1} \nameref{#1} }}
%     \begin{list}{\textbf{P\arabic{problem}\hfill}}{
%        \setlength{\listparindent}{0in}
%        \setlength{\labelwidth}{0.5in}
%        \setlength{\itemindent}{0in}
%        \setlength{\leftmargin}{0.65in}}
%    }
%    {\end{list}}

\newenvironment{review}[1]
    {\noindent \subsubsection*{#1}
     \begin{list}{\textbf{R\arabic{revproblem}\hfill}}{
        \setlength{\listparindent}{0in}
        \setlength{\labelwidth}{0.5in}
        \setlength{\itemindent}{0in}
        \setlength{\leftmargin}{0.65in}}
    }
    {\end{list}}

 \newwrite\solutions
 \immediate\openout\solutions=\jobname.sln
 \newcommand{\WritePLine}[2]{
    \immediate\write\solutions{\string\section*{#1 {\string\footnotesize (#2) \string\normalsize}}}
    \immediate\write\solutions{ \string\addcontentsline{toc}{section}{#2}}
    \immediate\write\solutions{ \string\begin{itemize}}
    \immediate\write\solutions{ \string\item[]}
    \immediate\write\solutions{ \string\InputIfFileExists{solutions/#2}{}{No Solution File Found}}
    \immediate\write\solutions{ \string\markboth{Solutions for Chapter \string\thechapter}{#1}}
    \immediate\write\solutions{ \string\end{itemize} \string\hrule}
    }

 \newcommand{\WriteChap}[1]{
    \immediate\write\solutions{\string\chapter*{#1}}
    \immediate\write\solutions{\string\addcontentsline{toc}{chapter}{#1}}
    }


\newcommand{\listproblemname}{}
\newlistof[chapter]{problem}{prb}{\listproblemname}

\newcommand{\prob}[1][prob:P\theproblem]
    {\phantomsection \refstepcounter{problem}\WritePLine{P\theproblem}{#1}
     \item[\textbf{P\theproblem}\hfill]\label{#1}}

\newcounter{subproblem}[problem]
\renewcommand{\thesubproblem}{\theproblem(\alph{subproblem})}
\renewcommand{\theHsubproblem}{\theproblem(\alph{subproblem})} %This is needed to keep unique links for hyperref
\newcommand{\subprob}{\refstepcounter{subproblem}\item[(\alph{subproblem})]}


\newcommand{\lab}[1]
    {\refstepcounter{problem}\WritePLine{L\theproblem}{#1}
     \item[\textbf{L\theproblem}\hfill]\label{#1}}

\newcounter{revprob}

\newcommand{\rev}[1]
    {\refstepcounter{revprob}\WritePLine{R\therevprob}{#1}
     \item[\textbf{R\therevprob}\hfill]\label{#1}}

\newenvironment{solution}
    {\par \rule{4.35in}{0.25pt} \nopagebreak \par
    \nopagebreak \scriptsize Solution: \nopagebreak   }
    {\par \nopagebreak \rule{4.35in}{0.25pt} \par \normalsize}

\newenvironment{answer}
    {\par \scriptsize Answer:}
    {\par \normalsize}



%%
% Set the font sizes and baselines to match Tufte's books
\renewcommand\normalsize{%
   \@setfontsize\normalsize\@xpt{14}%
   \abovedisplayskip 10\p@ \@plus2\p@ \@minus5\p@
   \abovedisplayshortskip \z@ \@plus3\p@
   \belowdisplayshortskip 6\p@ \@plus3\p@ \@minus3\p@
   \belowdisplayskip \abovedisplayskip
   \let\@listi\@listI}
\normalbaselineskip=14pt
\normalsize
\renewcommand\small{%
   \@setfontsize\normalsize\@xpt{14}%
   \abovedisplayskip 10\p@ \@plus2\p@ \@minus5\p@
   \abovedisplayshortskip \z@ \@plus3\p@
   \belowdisplayshortskip 6\p@ \@plus3\p@ \@minus3\p@
   \belowdisplayskip \abovedisplayskip
}
\renewcommand\footnotesize{%
   \@setfontsize\normalsize\@xpt{14}%
   \abovedisplayskip 10\p@ \@plus2\p@ \@minus5\p@
   \abovedisplayshortskip \z@ \@plus3\p@
   \belowdisplayshortskip 6\p@ \@plus3\p@ \@minus3\p@
   \belowdisplayskip \abovedisplayskip
}
\renewcommand\scriptsize{\@setfontsize\scriptsize\@viipt\@viiipt}
\renewcommand\tiny{\@setfontsize\tiny\@vpt\@vipt}
\renewcommand\large{\@setfontsize\large\@xipt{15}}
\renewcommand\Large{\@setfontsize\Large\@xiipt{16}}
\renewcommand\LARGE{\@setfontsize\LARGE\@xivpt{18}}
\renewcommand\huge{\@setfontsize\huge\@xxpt{30}}
\renewcommand\Huge{\@setfontsize\Huge{24}{36}}

\setlength\leftmargini   {1pc}
\setlength\leftmarginii  {1pc}
\setlength\leftmarginiii {1pc}
\setlength\leftmarginiv  {1pc}
\setlength\leftmarginv   {1pc}
\setlength\leftmarginvi  {1pc}
\setlength\labelsep      {.5pc}
\setlength\labelwidth    {\leftmargini}
\addtolength\labelwidth{-\labelsep}

%-----------------------------------------------------------
% Tool Label command
%-----------------------------------------------------------
\newcommand{\toollabel}[2]{%
   \protected@write \@auxout {}{\string \newlabel {#1}{{#2}{\thepage}{#2}{#1}{}} }%
   \hypertarget{#1}{#2}
}

%-----------------------------------------------------------
% math nodets for putting margin notes in equations.
% note: can't use these with equation numbers!
%-----------------------------------------------------------
\def\mathnote#1{%
  \tag*{\rlap{\hspace\marginparsep\smash{\parbox[t]{\marginparwidth}{%
  \footnotesize#1}}}}
}

%-----------------------------------------------------------
% Math shortcuts
%-----------------------------------------------------------
\newcommand{\abs}[1]{\left|{#1}\right|}
\newcommand{\ket}[1]{\left|{#1}\right\rangle}
\newcommand{\bra}[1]{\left\langle{#1}\right|}
\newcommand{\avg}[1]{\left\langle{#1}\right\rangle}
\newcommand{\oprod}[2]{\ket{#1}\mkern-6mu\bra{#2}}

\newcommand{\rmt}[1]{\textrm{#1}}

\newcommand{\com}[1]{\left[#1\right]}

\newcommand{\I}{i}
\newcommand{\E}[1]{e^{#1}}

\newcommand{\szmatrix}{\begin{pmatrix}1&0\\0&-1\end{pmatrix}}
\newcommand{\sxmatrix}{\begin{pmatrix}0&1\\1&0\end{pmatrix}}
\newcommand{\symatrix}{\begin{pmatrix}0&-\I\\\I&0\end{pmatrix}}

\newcommand{\onehat}{\hat{\mathbbm{1}}}

\newcommand{\Tr}{\rmt{Tr}}
\newcommand{\Pd}{{\cal P}}

\newcommand{\intii}{\int\displaylimits_{-\infty}^{\infty}}

\newcommand{\Ap}{\hat{\cal A}_{+}\mkern-1mu}
\newcommand{\Am}{\hat{\cal A}_{-}\mkern-1mu}
\newcommand{\Apm}{\hat{\cal A}_{\pm}\mkern-1mu}

\newcommand{\EAp}{\hat{\vphantom{\scalebox{1.1}{A}} \mathbbm{A}}_{+}}
\newcommand{\EAm}{\hat{\vphantom{\scalebox{1.1}{A}}\mathbbm{A}}_{-}}
\newcommand{\EApm}{\hat{\vphantom{\scalebox{1.1}{A}}\mathbbm{A}}_{\pm}}
\newcommand{\EN}{\hat{\mathbbm{N}}}



\newcommand{\Ja}{\rmt{\fontfamily{cmr}\selectfont \textit{J}}}
\newcommand{\Jh}{\skew{4}\hat{\Ja}}
\newcommand{\Jv}{\skew{4}\vec{\Ja}}
\newcommand{\Jp}{\Jh_{+}\mkern-1mu}
\newcommand{\Jm}{\Jh_{-}\mkern-1mu}
\newcommand{\Jpm}{\Jh_{\pm}\mkern-1mu}

\newcommand{\Ylm}{Y_l^{m_l}}

\renewcommand{\dagger}{\rmt{\fontfamily{cmr}\selectfont\textdagger}}

%-----------------------------------------------------------
% Math environment shortcuts
%-----------------------------------------------------------

\def\beq#1\eeq{\begin{equation}#1\end{equation}}
\def\bas#1\eas{\begin{subequations}\begin{align}#1\end{align}\end{subequations}}

\newcommand{\vket}[2]{\begin{pmatrix}#1\\#2\end{pmatrix}}
\newcommand{\vbra}[2]{\begin{pmatrix}#1 & #2\end{pmatrix}}

\newcommand{\Meq}{\Rightarrow}


\newcommand{\model}{{\bf Model:}\xspace}
\newcommand{\vis}{{\bf Visualization:}\xspace}
\newcommand{\sol}{{\bf Solution:}\xspace}
\newcommand{\assess}{{\bf Assess:}\xspace}


\newcommand{\hwp}{$\lambda/2$-plate\xspace}
\newcommand{\qwp}{$\lambda/4$-plate\xspace}
\newcommand{\stwo}{\sqrt{2}}





\newcommand{\highlight}[2][yellow]{\mathchoice%
  {\colorbox{#1}{$\displaystyle#2$}}%
  {\colorbox{#1}{$\textstyle#2$}}%
  {\colorbox{#1}{$\scriptstyle#2$}}%
  {\colorbox{#1}{$\scriptscriptstyle#2$}}}%

\definecolor{h1color}{rgb}{1,.8,.8}
\definecolor{h2color}{rgb}{.8,1,.8}
\definecolor{h3color}{rgb}{.8,.8,1}
\definecolor{h4color}{rgb}{1,1,.8}



%-----------------------------------------------------------
% Tikz Drawing shortcuts
%-----------------------------------------------------------


\newcommand{\detector}[4]
{
  % this needed to be modified somehow...
  \begin{scope}[scale=#1,shift={(#2,#3)}]
      \begin{scope}[rotate around={#4:(0,0)}]
      \filldraw[fill=black!20] (0,.5) -- (0,-.5) -- (0,-.5) arc(-90:90:.5);
      \draw[thick] plot[smooth] coordinates{(0.5,0)(.75,.2)(1.0,-.2)(1.5,0)};
      \end{scope}
  \end{scope}
}


\newcommand\pgfmathsinandcos[3]{%
  \pgfmathsetmacro#1{sin(#3)}%
  \pgfmathsetmacro#2{cos(#3)}%
}
\newcommand\LongitudePlane[3][current plane]{%
  \pgfmathsinandcos\sinEl\cosEl{#2} % elevation
  \pgfmathsinandcos\sint\cost{#3} % azimuth
  \tikzset{#1/.style={cm={\cost,\sint*\sinEl,0,\cosEl,(0,0)}}}
}
\newcommand\LatitudePlane[3][current plane]{%
  \pgfmathsinandcos\sinEl\cosEl{#2} % elevation
  \pgfmathsinandcos\sint\cost{#3} % latitude
  \pgfmathsetmacro\yshift{\cosEl*\sint}
  \tikzset{#1/.style={cm={\cost,0,0,\cost*\sinEl,(0,\yshift)}}} %
}
\newcommand\DrawLongitudeCircle[2][1]{
  \LongitudePlane{\angEl}{#2}
  \tikzset{current plane/.prefix style={scale=#1}}
   % angle of "visibility"
  \pgfmathsetmacro\angVis{atan(sin(#2)*cos(\angEl)/sin(\angEl))} %
  \draw[current plane] (\angVis:1) arc (\angVis:\angVis+180:1);
  \draw[current plane,dashed] (\angVis-180:1) arc (\angVis-180:\angVis:1);
}
\newcommand\DrawLatitudeCircle[2][1]{
  \LatitudePlane{\angEl}{#2}
  \tikzset{current plane/.prefix style={scale=#1}}
  \pgfmathsetmacro\sinVis{sin(#2)/cos(#2)*sin(\angEl)/cos(\angEl)}
  % angle of "visibility"
  \pgfmathsetmacro\angVis{asin(min(1,max(\sinVis,-1)))}
  \draw[current plane] (\angVis:1) arc (\angVis:-\angVis-180:1);
  \draw[current plane,dashed] (180-\angVis:1) arc (180-\angVis:\angVis:1);
}




%sagemathcloud={"latex_command":""}